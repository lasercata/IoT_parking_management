\documentclass[a4paper, 11pt]{article}


%------------------------------------------------------------------------
%
% Author                :   Flow, Lasercata
% Last modification     :   2026.01.30
%
%------------------------------------------------------------------------

%---------Init {{{1
%------Lang
%\usepackage[french]{babel}
%\usepackage[english]{babel}


%See https://github.com/lasercata/LaTeX_Templates for the file latex_base.sty
\input{style/latex_base.sty}

% \newcommand{\Names}{
    Name1 \textsc{Last-Name1} \\
    Name2 \textsc{Last-Name2}
}


\input{data/data.sty}

\renewcommand{\arraystretch}{1.4}
%}}}1

%------Title (with default LaTeX style)
\title{IoT-Based smart parking system for disabled people}
\author{
    \Names
}
\date{\today}

%---------------------------------Begin Document
\begin{document}
    
    % Title {{{1
    % \thetitle{}{}
    \maketitle
    
    \setcounter{tocdepth}{2} % hide subsubsections from the table of contents
    \tableofcontents
    %\listoffigures
    \listoftables
    %\listofalgorithms
    %\lstlistoflistings
    \newpage
    % }}}1
    
    \begin{indt}{\section{Chosen Scenario}}% {{{1
        This project focuses on the design of an IoT-based smart parking system dedicated to people with disabilities. The objective is to ensure fair usage of reserved parking spaces, prevent fraud related to authorization cloning, and improve accessibility by providing real-time availability and booking capabilities.

        Each parking space is equipped with an IoT node capable of detecting vehicle presence and verifying the validity of a disabled parking authorization. The system continuously monitors parking activity and communicates with a centralized backend that manages authorizations, detects anomalies, and informs law enforcement when violations are detected.

        In addition, an interface allows authorized users to visualize nearby available parking spaces, reserve a spot up to one hour in advance, and receive feedback on their parking session. 
    \end{indt}% }}}1

    \begin{indt}{\section{Functional Requirements}} %{{{1
        \begin{table}[H]% {{{2
            \centering
        
            \begin{tabular}{|l p{350pt}|}
                \hline
                \multicolumn{2}{|c|}{
                    \textbf{FR1}
                    --- Occupancy detection
                }
                \\
                \hline
                \hline
                \textbf{Description}
                & The solution is able to detect the presence of a vehicle on a disabled parking space.
                \\
                \hline
                \textbf{Input}
                & Sensor data
                \\
                \textbf{Output}
                & Occupancy status (occupied / free)
                \\
                \hline
            \end{tabular}
        
            \caption{\textbf{FR1} --- Occupancy detection}
            \label{tab:fr1}
        \end{table}% }}}2

        \begin{table}[H]% {{{2
            \centering
        
            \begin{tabular}{|l p{350pt}|}
                \hline
                \multicolumn{2}{|c|}{
                    \textbf{FR2} --- Authorization verification
                }
                \\
                \hline
                \hline
                \textbf{Description}
                & The solution is able to verify the validity of a disabled parking authorization upon vehicle detection.
                \\
                \hline
                \textbf{Input}
                & Vehicle identifier, authorization ID
                \\
                \textbf{Output}
                & Authorization status (valid / invalid / expired)
                \\
                \hline
            \end{tabular}
        
            \caption{\textbf{FR2} --- Authorization verification}
            \label{tab:fr2}
        \end{table}% }}}2

        \begin{table}[H]% {{{2
            \centering
        
            \begin{tabular}{|l p{350pt}|}
                \hline
                \multicolumn{2}{|c|}{
                    \textbf{FR3} --- Authorization cloning detection
                }
                \\
                \hline
                \hline
                \textbf{Description}
                & The solution is able to identify simultaneous usage of the same authorization on geographically distinct parking spaces.
                \\
                \hline
                \textbf{Input}
                & Authorization ID, GPS location, timestamps
                \\
                \textbf{Output}
                & Cloning alert / None
                \\
                \hline
            \end{tabular}
        
            \caption{\textbf{FR3} --- Authorization cloning detection}
            \label{tab:fr3}
        \end{table}% }}}2

        \begin{table}[H]% {{{2
            \centering
        
            \begin{tabular}{|l p{350pt}|}
                \hline
                \multicolumn{2}{|c|}{
                    \textbf{FR4} --- Violation identification
                }
                \\
                \hline
                \hline
                \textbf{Description}
                & The solution is able to identify unauthorized parking, expired authorizations, or missing authorizations.
                \\
                \hline
                \textbf{Input}
                & Occupancy status, authorization status
                \\
                \textbf{Output}
                & Violation flag
                \\
                \hline
            \end{tabular}
        
            \caption{\textbf{FR4} --- Violation identification}
            \label{tab:fr4}
        \end{table}% }}}2

        \begin{table}[H]% {{{2
            \centering
        
            \begin{tabular}{|l p{350pt}|}
                \hline
                \multicolumn{2}{|c|}{
                    \textbf{FR5} --- Authority notification
                }
                \\
                \hline
                \hline
                \textbf{Description}
                & The solution is able to notify law enforcement authorities in real time when a violation or cloning event is detected.
                \\
                \hline
                \textbf{Input}
                & Violation flag or cloning alert 
                \\
                \textbf{Output}
                & Notification message sent to authorities
                \\
                \hline
            \end{tabular}
        
            \caption{\textbf{FR5} --- Authority notification}
            \label{tab:fr5}
        \end{table}% }}}2

        \begin{table}[H]% {{{2
            \centering
        
            \begin{tabular}{|l p{350pt}|}
                \hline
                \multicolumn{2}{|c|}{
                    \textbf{FR6} --- Parking availability visualization
                }
                \\
                \hline
                \hline
                \textbf{Description}
                & The solution is able to provide real-time availability of disabled parking spaces within a selected geographical area.
                \\
                \hline
                \textbf{Input}
                & Parking spot status data
                \\
                \textbf{Output}
                & Availability map / list
                \\
                \hline
            \end{tabular}
        
            \caption{\textbf{FR6} --- Parking availability visualization}
            \label{tab:fr6}
        \end{table}% }}}2

        \begin{table}[H]% {{{2
            \centering
        
            \begin{tabular}{|l p{350pt}|}
                \hline
                \multicolumn{2}{|c|}{
                    \textbf{FR7} --- Parking reservation
                }
                \\
                \hline
                \hline
                \textbf{Description}
                & The user is able to reserve a parking space up to one hour in advance.
                \\
                \hline
                \textbf{Input}
                & User request, parking availability
                \\
                \textbf{Output}
                & Reservation confirmation
                \\
                \hline
            \end{tabular}
        
            \caption{\textbf{FR7} --- Parking reservation}
            \label{tab:fr7}
        \end{table}% }}}2

        \begin{table}[H]% {{{2
            \centering
        
            \begin{tabular}{|l p{350pt}|}
                \hline
                \multicolumn{2}{|c|}{
                    \textbf{FR8} --- Reservation timeout management
                }
                \\
                \hline
                \hline
                \textbf{Description}
                & The solution is able to automatically cancel reservations if the vehicle does not arrive within a predefined grace period.
                \\
                \hline
                \textbf{Input}
                & Reservation timestamp, occupancy status
                \\
                \textbf{Output}
                & Reservation cancellation notice (to the user), occupancy status set to free
                \\
                \hline
            \end{tabular}
        
            \caption{\textbf{FR8} --- Reservation timeout management}
            \label{tab:fr8}
        \end{table}% }}}2

        \begin{table}[H]% {{{2
            \centering
        
            \begin{tabular}{|l p{350pt}|}
                \hline
                \multicolumn{2}{|c|}{
                    \textbf{FR9} --- Parking occupancy status
                }
                \\
                \hline
                \hline
                \textbf{Description}
                & The solution is able to display its current status (free, reserved, parked)
                \\
                \hline
                \textbf{Input}
                & Sensor data, reservation data
                \\
                \textbf{Output}
                & Status indicator (free / reserved / parked)
                \\
                \hline
            \end{tabular}
        
            \caption{\textbf{FR9} --- Parking occupancy status}
            \label{tab:fr9}
        \end{table}% }}}2

        \begin{table}[H]% {{{2
            \centering
        
            \begin{tabular}{|l p{350pt}|}
                \hline
                \multicolumn{2}{|c|}{
                    \textbf{FR10} --- User registration
                }
                \\
                \hline
                \hline
                \textbf{Description}
                & The user/admin is able to register to the platform
                \\
                \hline
                \textbf{Input}
                & User registration data
                \\
                \textbf{Output}
                & User account created
                \\
                \hline
            \end{tabular}
        
            \caption{\textbf{FR10} --- User registration}
            \label{tab:fr10}
        \end{table}% }}}2
    \end{indt} %}}}1

    \begin{indt}{\section{Non-Functional Requirements}} %{{{1
        \begin{table}[H]
            \centering

            \begin{tabular}{p{2cm} p{4cm} p{8cm}}
                \hline

                \textbf{ID}
                & \textbf{Category}
                & \textbf{Requirement}
                \\

                \hline
                NFR1
                & Performance
                & It is possible to complete occupancy detection and authorization verification within 10 seconds after vehicle arrival.
                \\

                NFR2
                & Reliability
                & The system is available at least 99\% of the time and tolerant to temporary network outages.
                \\

                NFR3
                & Durability
                & The IoT nodes are reliable in outdoor environments with temperatures ranging from -20\textdegree C to 60\textdegree C.
                \\

                NFR4
                & Portability
                & The system is portable, allowing new parking nodes to be integrated without modifying existing nodes or backend services.
                \\

                NFR5
                & Resource efficiency
                & The parking node energy consumption is below 100 mW on average and supports low-power operating modes.
                \\

                NFR6
                & Security
                & The system must be secure from basic cybersecurity threats (\textit{e.g} communications are encrypted)
                \\

                NFR7
                & Privacy
                & The system must limit the amount of personal data processed, and limit the access of needed data (\textit{e.g} identifiers are anonymized).
                \\

                NFR8
                & Affordability
                & The hardware cost per parking space and the software maintenance cost is compatible with large-scale municipal deployment.
                \\

                NFR9
                & Compatibility
                & The application must be compatible with end users' devices
                \\

                \hline
            \end{tabular}

            \caption{Non-Functional Requirements}
            \label{tab:nfr}
        \end{table}
    \end{indt} %}}}1

    \begin{indt}{\section{Analysis}} %{{{1
        \begin{indt}{\subsection{Board selection}} %{{{2
            TODO (this is a global choice)

            Node MCU ESP8266, Arduino UNO, Raspberry Pi 3
        \end{indt} %}}}2

        \begin{indt}{\subsection{User interface}} %{{{2
            \begin{indt}{\subsubsection{Introduction}} %{{{3
                To address the functional requirements \textbf{FR6} (parking availability visualization), \textbf{FR7} (advance reservation), and \textbf{FR10} (user registration), a \emph{user interface} is needed.
                
                The user interface must support role differentiation (user, regulator, administrator) and give different permissions to each.

                It must also enable administrators to list all the nodes and monitor their status.
                
                \begin{indi}{Possible options are as follows:}
                    \item Native phone app
                    \item Web app
                    \item Command line tool (can be in the form of a Discord/Telegram bot)
                \end{indi}
            \end{indt} %}}}3

            \begin{indt}{\subsubsection{Comparison of user interfaces}} %{{{3
                \begin{table}[H]% {{{4
                    \centering

                    {\small
                    \begin{tabular}{p{3cm} p{80pt} p{80pt} p{80pt} p{80pt}}
                        \hline
                        \textbf{Technology}
                        & \textbf{Reliability (NF2)}
                        & \textbf{Privacy (NF7)}
                        & \textbf{Afforda\-bility (NF8)}
                        & \textbf{Compati\-bility (NF9)}
                        \\

                        \hline
                        Phone app
                        & \cellcolor{pastelOrange} Depends on network, but can cache data
                        & \cellcolor{pastelGreen} Good privacy is possible
                        & \cellcolor{pastelRed} Need to maintain multiple app versions
                        & \cellcolor{pastelRed} Users need compatible phone
                        \\

                        Web app
                        & \cellcolor{pastelRed} Depends on network's availability
                        & \cellcolor{pastelGreen} Good privacy is possible
                        & \cellcolor{pastelGreen} Low maintenance cost
                        & \cellcolor{pastelGreen} Most users' devices are able to interact with web pages
                        \\

                        Command line app
                        & \cellcolor{pastelOrange} Depends on network, but can cache data
                        & \cellcolor{pastelOrange} If using Discord/Telegram, data is shared to that third-party
                        & \cellcolor{pastelGreen} Low maintenance cost
                        & \cellcolor{pastelOrange} While probably compatible with user's devices, it is not user-friendly
                        \\

                        \hline
                    \end{tabular}
                    }

                    \caption{Comparison of user interfaces}
                    \label{tab:UI_comparison}
                \end{table}% }}}4
            \end{indt} %}}}3

            \begin{indt}{\subsubsection{Conclusion}} %{{{3
                In the light of the \autoref{tab:UI_comparison}, the \emph{web interface} seems the most appropriate for the user interface as it has the largest compatibility.
                Regarding reliability, in the event of a network outage, no real time data is possible. Even if with a native application, it might possible to cache data, it will become outdated and the visualisation will be wrong.
                We put a stronger emphasis on the compatibility of end user's device.

                This interface will also be appropriate to implement the multi-level authorizations (admin, user, ...)
            \end{indt} %}}}3
        \end{indt} %}}}2

        \begin{indt}{\subsection{Network protocols}} %{{{2
            To fulfil the functional requirements \textbf{FR5} (authority notification), \textbf{FR6} (Real time parking visualization), \textbf{FR7} (advance reservation), there need to be communication between the different entities (the \emph{nodes}, the \emph{IoT platform}, and the \emph{user interface} (UI)).

            \begin{indi}{Based on the requirements, there are three types of communications:}
                \item[(1)] platform $\longleftarrow$ UI (visualization, reservation)
                \item[(2)] platform $\longrightarrow$ UI (authority notification)
                \item[(3)] nodes $\longrightarrow$ platform (status update for visualization, violation notification)
                \item[(4)] nodes $\longleftarrow$ platform (reservation)
            \end{indi}

            \vspace{6pt}
            
            $\bullet$ The case (2) is addressed in the \autoref{sub:fr5}.

            \vspace{6pt}
            
            $\bullet$ For the case (1), the only relevant option is to use the HTTP client/server protocol, with the platform being the server.

            \vspace{6pt}
            
            $\bullet$ For (3), the update must be fast in order to have a \emph{real-time} visualization. Here again, the client/server model seems more appropriate: the nodes are clients that connect to the server (the platform) when something occurs. It provides a real-time update and it light for the nodes (that have limited resources).

            \vspace{6pt}
            
            $\bullet$ Remain the case (4). Here the platform has to send data to a given node.

            \begin{indi}{The following approaches are possible:}
                \item[(A)] client/server, each node implements a server ;
                \item[(B)] client/server, platform is the server and the nodes regularly check for update ;
                \item[(C)] publish-subscriber, each node subscribes to a topic with its ID so that the platform can publish to a given node.
            \end{indi}

            \begin{table}[H]% {{{3
                \centering

                {\small
                \begin{tabular}{p{3cm} p{80pt} p{80pt} p{80pt} p{80pt}}
                    \hline
                    \textbf{Protocol}
                    & \textbf{Performance (NF1)}
                    & \textbf{Reliability (NF2)}
                    & \textbf{Portability (NF4)}
                    & \textbf{Resource efficiency (NF5)}
                    \\

                    \hline
                    (A)
                    & \cellcolor{pastelGreen} Good performance
                    & \cellcolor{pastelOrange} Each node is independent, but depend on the platform
                    & \cellcolor{pastelRed} Need to update the platform when adding new nodes
                    & \cellcolor{pastelRed} Each node being a server consumes a lot
                    \\

                    (B)
                    & \cellcolor{pastelOrange} Depends on the frequency of requests
                    & \cellcolor{pastelOrange} Each node is independent, but depend on the platform
                    & \cellcolor{pastelGreen} Adding/removing nodes is transparent
                    & \cellcolor{pastelRed} Generate a lot of network traffic
                    \\

                    (C)
                    & \cellcolor{pastelGreen} Good performance
                    & \cellcolor{pastelOrange} Each node is independent, but depend on the platform
                    & \cellcolor{pastelGreen} Adding/removing nodes is transparent
                    & \cellcolor{pastelOrange} Generally lower power needed
                    \\

                    \hline
                \end{tabular}
                }

                \caption{Comparison of protocols for Platform $\longrightarrow$ node communication}
                \label{tab:protocols_comparison}
            \end{table}% }}}3

            According to \autoref{tab:protocols_comparison}, the publish-subscriber solution seems the most appropriated approach for this communication.
        \end{indt} %}}}2

        \begin{indt}{\subsection{FR1 -- Occupancy detection}} %{{{2
            \begin{indt}{\subsubsection{Introduction}} %{{{3
                The objective of this functional requirement is to detect the presence of a vehicle occupying a disabled parking space. This functionality constitutes the foundation of the smart parking system, as it triggers all subsequent processes such as authorization verification, fraud detection, and violation reporting.
                
                Within the context of an IoT-based parking system dedicated to people with disabilities, occupancy detection must be reliable, accurate, and continuously operational in outdoor urban environments. Any detection failure would directly impact the system’s ability to ensure fair usage of reserved parking spaces and to prevent misuse or fraud.
            \end{indt} %}}}3

            \begin{indt}{\subsubsection{Comparison of occupancy detection technologies}} %{{{3
                Several sensing technologies can be considered to detect vehicle presence. \autoref{tab:fr1_comparison} presents a comparison of the most relevant options for this project.

                \begin{table}[H]% {{{4
                    \centering

                    \begin{tabular}{p{3cm} p{2cm} p{2cm} p{2cm} p{2cm} p{2cm} p{2cm}}
                        \hline
                        \textbf{Technology}
                        & \textbf{Performan\-ce (NF1)}
                        & \textbf{Reliability (NF2)}
                        & \textbf{Durability (NF3)}
                        & \textbf{Resource efficiency (NF5)}
                        & \textbf{Privacy (NF7)}
                        & \textbf{Affordability (NF8)}
                        \\

                        \hline
                        Ultrasonic sensor
                        \newline
                        \texttt{HC-SR04}
                        & \cellcolor{white} detection time?
                        & \cellcolor{pastelOrange} Sensitive to rain, snow, and sensor orientation
                        & \cellcolor{white} ?
                        & \cellcolor{pastelGreen} Low power consumption
                        & \cellcolor{pastelGreen} No personal data
                        & \cellcolor{pastelGreen} Low cost
                        \\

                        Magnetic ground sensor
                        \newline
                        \texttt{Honeywell HMC5883L}
                        & \cellcolor{white} detection time?
                        & \cellcolor{pastelGreen} High
                        & \cellcolor{pastelGreen} High
                        & \cellcolor{white} ?
                        & \cellcolor{pastelGreen} No personal data
                        & \cellcolor{pastelRed} Requires ground installation
                        \\

                        Camera-based detection
                        \newline
                        \texttt{Raspberry Pi Camera Module v2}
                        & \cellcolor{white} detection time?
                        & \cellcolor{pastelGreen} High accuracy
                        & \cellcolor{pastelGreen} High
                        & \cellcolor{pastelRed} Hight power consumption
                        & \cellcolor{pastelRed} Processing of personal data
                        & \cellcolor{pastelOrange}?
                        \\

                        Infrared sensor
                        \newline
                        \texttt{Panasonic EKMB Series PIR}
                        & \cellcolor{white} detection time?
                        & \cellcolor{pastelRed} affected by ambient temperature and sunlight
                        & \cellcolor{pastelGreen} High
                        & \cellcolor{pastelGreen} Low power consumption
                        & \cellcolor{pastelGreen} No personal data
                        & \cellcolor{pastelGreen} ?
                        \\

                        \hline
                    \end{tabular}

                    \caption{Comparison of Occupancy Detection Solutions}
                    \label{tab:fr1_comparison}
                \end{table}% }}}4
            \end{indt} %}}}3

            \begin{indt}{\subsubsection{Conclusion}} %{{{3
                The magnetic ground sensor is selected as the most appropriate solution for occupancy detection in this project. Its robustness against environmental conditions such as rain, lighting variations, and temperature changes makes it particularly well suited for outdoor deployment in urban parking spaces. In addition, magnetic sensing does not involve the collection of visual data, thereby avoiding privacy issues associated with camera-based approaches.

                Although installation requires embedding the sensor into the ground, this constraint is acceptable for a long-term infrastructure dedicated to disabled parking spaces. The high detection reliability, low false positive rate, and compatibility with low-power IoT nodes make magnetic ground sensors the best choice for fulfilling FR1.
            \end{indt} %}}}3

            \begin{indt}{\subsubsection{References}} %{{{3
                \begin{indi}[0pt]{}
                    \item HC-SR04 Ultrasonic Sensor Datasheet: \\
                        \url{https://www.handsontec.com/dataspecs/HC-SR04-Ultrasonic.pdf}
                    \item Honeywell HMC5883L Magnetometer Datasheet: \\
                        \url{https://www.farnell.com/datasheets/1683374.pdf}
                    \item Raspberry Pi Camera Module v2 Documentation: \\
                        \url{https://www.raspberrypi.com/documentation/accessories/camera.html}
                    \item Panasonic EKMB Series PIR Sensor Datasheet: \\
                        \url{https://na.industrial.panasonic.com/products/sensors/sensors-automotive-industrial-applications/lineup/pir-motion-sensor-papirs/series/149462}
                \end{indi}
            \end{indt} %}}}3
        \end{indt} %}}}2

        \begin{indt}{\subsection{FR2 -- Authorization verification}} %{{{2
            \label{sub:fr2}

            \begin{indt}{\subsubsection{Introduction}} %{{{3
                The purpose of this functional requirement is to verify the validity of a disabled parking authorization when a vehicle is detected occupying a reserved parking space. This step is essential to distinguish between legitimate users and unauthorized vehicles, ensuring that disabled parking spaces are used fairly and in accordance with regulations.

                In the context of the proposed IoT-based smart parking system, authorization verification must be reliable, secure, and fast. Once occupancy is detected, the system must be able to associate the vehicle with a valid authorization and determine whether this authorization is authentic and still valid. This functionality directly contributes to fraud prevention, particularly against expired permits or stolen identifiers, and forms the basis for subsequent violation detection and enforcement actions.
            \end{indt} %}}}3

            \begin{indt}{\subsubsection{Comparison of authorization verification technologies}} %{{{3
                Several technical approaches can be used to identify vehicles and verify disabled parking authorizations. \autoref{tab:fr2_comparison} compares the most relevant solutions for this project.

                \begin{indi}{Possible technologies:}
                    \item \textbf{RFID}: the user carries a RFID tag and present it to the node

                    \item \textbf{Bluetooth}: the user uses a phone application with the bluetooth activated

                    \item \textbf{License plate recognition}: the node reads the car's plate using a camera

                    \item \textbf{QR code on the node}: there is a QR code printed on each node (different for each) and the user scans it with a phone application

                    \item \textbf{QR code reader}: through its interface, the user generates a QR code that is read by the node
                \end{indi}

                \begin{table}[H]% {{{4
                    \centering

                    {\small
                    \begin{tabular}{p{3cm} p{50pt} p{50pt} p{50pt} p{50pt} p{50pt} p{50pt} p{50pt}}
                        \hline
                        \textbf{Technology}
                        & \textbf{Performan\-ce (NF1)}
                        & \textbf{Reliability (NF2)}
                        & \textbf{Durability (NF3)}
                        & \textbf{Resource efficiency (NF5)}
                        & \textbf{Privacy (NF7)}
                        & \textbf{Afforda\-bi\-lity (NF8)}
                        & \textbf{Compati\-bility (NF9)}
                        \\

                        \hline
                        RFID
                        \newline
                        \texttt{MFRC522 RFID Reader}
                        & \cellcolor{pastelGreen} Reading time < 1s
                        & \cellcolor{pastelGreen} High
                        & \cellcolor{pastelGreen} High
                        & \cellcolor{pastelGreen} Low power consumption
                        & \cellcolor{pastelGreen} No personal data
                        & \cellcolor{pastelGreen} Low cost
                        & \cellcolor{pastelGreen} User must carry a tag
                        \\

                        Bluetooth Low Energy (BLE)
                        \newline
                        \texttt{ESP32 BLE Module}
                        & \cellcolor{white} Reading / connection time?
                        & \cellcolor{pastelGreen} High
                        & \cellcolor{pastelGreen} High
                        & \cellcolor{white} ?
                        & \cellcolor{pastelOrange} Identifiable data exchanged (Bluetooth mac)
                        & \cellcolor{white} ?
                        & \cellcolor{pastelRed} User must have a phone compatible with the application (1)
                        \\

                        License Plate Recognition (LPR)
                        \newline
                        \texttt{Raspberry Pi + Camera Module v2}
                        & \cellcolor{white} Reading time?
                        & \cellcolor{pastelOrange} Good
                        & \cellcolor{pastelGreen} High
                        & \cellcolor{pastelRed} High power consumption
                        & \cellcolor{pastelRed} Personal data processed
                        & \cellcolor{white} ?
                        & \cellcolor{pastelGreen} No user interaction
                        \\

                        QR code
                        \newline
                        \texttt{QR code on the node}
                        & \cellcolor{white} Reading time?
                        & \cellcolor{pastelGreen} High
                        & \cellcolor{pastelGreen} Good
                        & \cellcolor{pastelGreen} No power consumption
                        & \cellcolor{pastelGreen} No personal data processed
                        & \cellcolor{pastelGreen} Low cost
                        & \cellcolor{pastelRed} User must have a phone compatible with the application (1)
                        \\

                        QR code reader
                        \newline
                        \texttt{QR code reader}
                        & \cellcolor{white} Reading time?
                        & \cellcolor{pastelGreen} High
                        & \cellcolor{pastelGreen} ?
                        & \cellcolor{pastelOrange} ?
                        & \cellcolor{pastelGreen} No personal data processed
                        & \cellcolor{white} ?
                        & \cellcolor{pastelOrange} User must have an internet connection
                        \\

                        \hline
                    \end{tabular}
                    }

                    \caption{Comparison of Authorization Verification Solutions}
                    \label{tab:fr2_comparison}
                \end{table}% }}}4

                (1): This requires to develop and maintain applications for android and iOS, across multiple versions of the OSes, and requires that all users have a phone.

                % TODO: reliability (NF2) here does not take into account the network (in fact, for security, it should always check the validity from the server.)
            \end{indt} %}}}3

            \begin{indt}{\subsubsection{Conclusion}} %{{{3
                % TODO: change this.

                Bluetooth Low Energy (BLE)–based authorization verification is selected as the most suitable solution for this project. BLE enables automatic and contactless identification of authorized users through a secure identifier broadcast by a registered device, such as a smartphone or a dedicated BLE tag. This approach balances ease of use for people with disabilities and technical feasibility for an IoT-based infrastructure.

                Compared to RFID, BLE offers a longer communication range and greater flexibility, while avoiding the privacy and computational complexity associated with camera-based license plate recognition. Furthermore, BLE is natively supported by many low-power microcontrollers, facilitating seamless integration with the parking space IoT nodes. As a result, BLE-based verification best satisfies the requirements of reliability, scalability, and user accessibility for FR2.
            \end{indt} %}}}3

            \begin{indt}{\subsubsection{References}} %{{{3
                \begin{indi}[0pt]{}
                    \item MFRC522 RFID Reader Datasheet: \\
                        \url{https://www.nxp.com/docs/en/data-sheet/MFRC522.pdf}
                    \item ESP32 Technical Reference Manual (BLE): \\
                        \url{https://www.espressif.com/sites/default/files/documentation/esp32_technical_reference_manual_en.pdf}
                    \item Raspberry Pi Camera Module v2 Documentation: \\
                        \url{https://www.raspberrypi.com/documentation/accessories/camera.html}
                \end{indi}
            \end{indt} %}}}3
        \end{indt} %}}}2

        \begin{indt}{\subsection{FR3 -- Authorization cloning detection}} %{{{2
            \begin{indt}{\subsubsection{Introduction}} %{{{3
                The objective of this functional requirement is to detect the simultaneous use of the same disabled parking authorization across distinct parking spaces.
                Authorization cloning represents a significant form of fraud, where a single valid identifier is duplicated or shared in order to illegally occupy multiple reserved spaces at the same time.

                This functionality is essential to ensure fairness, preserve trust in the system, and support law enforcement by automatically flagging suspicious behaviour.

                \vspace{12pt}

                Independently of the choice of the authorization technology (\textit{cf} \autoref{sub:fr2}), the authorization will be based on some kind of identifier (ID).
                \begin{indi}{But the type of the ID varies:}
                    \item In the case of the \textbf{QR code on the node}, the ID identifies the node itself. Here, authorization cloning means that the same user account scans multiple distinct nodes.

                    \item In the other cases, the ID represent the user. Cloning here means that multiple distinct nodes read the same user ID.

                    \item For the \textbf{QR code on the node} and for the \textbf{license plate reader}, the ID is fixed and not customizable.
                \end{indi}
                
                \begin{indi}{So to address this requirement, multiple options are possible:}
                    \item If the ID represents the user, it will be retrieved by the node. In this case, to detect cloning, the node can either ask all the other nodes if the same ID is in use, or ask the IoT platform.
                        Otherwise, if the ID represent the node, it is only possible to address this requirement in the IoT platform.

                    \item If the ID represents the user and is customizable, it is possible to include in it a random part, modified/generated at each reading, that will be used for authentication (like a one time code), thus making it impossible to have two valid authorisations at the same time.
                \end{indi}
            \end{indt} %}}}3

            \begin{indt}{\subsubsection{Comparison of cloning detection approaches}} %{{{3
                \begin{table}[H]% {{{4
                    \centering

                    {\small
                    \begin{tabular}{p{3cm} p{100pt} p{100pt} p{100pt}}
                        \hline
                        \textbf{Approach}
                        & \textbf{Performan\-ce (NF1)}
                        & \textbf{Reliability (NF2)}
                        & \textbf{Resource efficiency (NF5)}
                        \\

                        \hline
                        Detection at the \emph{nodes} level
                        & \cellcolor{pastelRed} A lot of network overhead
                        & \cellcolor{pastelOrange} If a node's network is down, it will not answer
                        & \cellcolor{pastelRed} Due to the numerous network requests, high energy consumption
                        \\

                        Detection at the \emph{Platform} level
                        & \cellcolor{pastelGreen} Good performance (one network request)
                        & \cellcolor{pastelOrange} needs connection to the Platform
                        & \cellcolor{pastelGreen} low consumption
                        \\

                        \hline
                    \end{tabular}
                    }

                    \caption{Comparison of Authorization Cloning Detection Approaches}
                    \label{tab:fr3_comparison}
                \end{table}% }}}4
            \end{indt} %}}}3

            \begin{indt}{\subsubsection{Conclusion}} %{{{3
                In the light of the \autoref{tab:fr3_comparison}, the detection seems more advantageous at the \emph{Platform} level.

                Furthermore, by correlating authorization identifiers with spatial and temporal data received from distributed parking nodes, the system can identify inconsistencies that are physically impossible under normal usage conditions.
            \end{indt} %}}}3
        \end{indt} %}}}2

        \begin{indt}{\subsection{FR4 -- Violation identification}} %{{{2
            \begin{indt}{\subsubsection{Introduction}} %{{{3
                The objective of this functional requirement is to identify parking violations related to disabled parking spaces, including unauthorized parking, expired authorizations, or the absence of any valid authorization. This functionality represents a critical decision layer within the smart parking system, as it transforms raw detection and verification data into actionable enforcement events.
                
                In the context of an IoT-based smart parking system, violation identification relies on the correlation of occupancy detection results with authorization verification outcomes. By applying predefined decision rules, the system can automatically and consistently determine whether a parking event complies with regulations. This automation reduces the need for manual inspection, improves fairness, and ensures that violations are detected in real time.
            \end{indt} %}}}3

            % \begin{indt}{\subsubsection{Comparison of violation identification mechanisms}} %{{{3
            %     Several approaches can be used to identify parking violations based on system inputs. \autoref{tab:fr4_comparison} presents a comparison of the most relevant mechanisms for this project.
            %
            %     \begin{table}[H]% {{{4
            %         \centering
            %
            %         {\small
            %         \begin{tabular}{p{3cm} p{50pt} p{50pt} p{50pt} p{50pt} p{50pt} p{50pt} p{50pt}}
            %             \hline
            %             \textbf{Technology}
            %             & \textbf{Performan\-ce (NF1)}
            %             & \textbf{Reliability (NF2)}
            %             & \textbf{Durability (NF3)}
            %             & \textbf{Resource efficiency (NF5)}
            %             & \textbf{Privacy (NF7)}
            %             & \textbf{Afforda\-bi\-lity (NF8)}
            %             & \textbf{Compati\-bility (NF9)}
            %             \\
            %
            %             \hline
            %             XXX
            %             \newline
            %             \texttt{ref}
            %             & \cellcolor{white} XXX
            %             & \cellcolor{white} XXX
            %             & \cellcolor{white} XXX
            %             & \cellcolor{white} XXX
            %             & \cellcolor{white} XXX
            %             & \cellcolor{white} XXX
            %             & \cellcolor{white} XXX
            %             \\
            %
            %             XXX
            %             \newline
            %             \texttt{ref}
            %             & \cellcolor{white} XXX
            %             & \cellcolor{white} XXX
            %             & \cellcolor{white} XXX
            %             & \cellcolor{white} XXX
            %             & \cellcolor{white} XXX
            %             & \cellcolor{white} XXX
            %             & \cellcolor{white} XXX
            %             \\
            %
            %             XXX
            %             \newline
            %             \texttt{ref}
            %             & \cellcolor{white} XXX
            %             & \cellcolor{white} XXX
            %             & \cellcolor{white} XXX
            %             & \cellcolor{white} XXX
            %             & \cellcolor{white} XXX
            %             & \cellcolor{white} XXX
            %             & \cellcolor{white} XXX
            %             \\
            %
            %             XXX
            %             \newline
            %             \texttt{ref}
            %             & \cellcolor{white} XXX
            %             & \cellcolor{white} XXX
            %             & \cellcolor{white} XXX
            %             & \cellcolor{white} XXX
            %             & \cellcolor{white} XXX
            %             & \cellcolor{white} XXX
            %             & \cellcolor{white} XXX
            %             \\
            %
            %             \hline
            %         \end{tabular}
            %         }
            %
            %         \caption{Comparison of violation identification mechanisms}
            %         \label{tab:fr4_comparison}
            %     \end{table}% }}}4
            % \end{indt} %}}}3
            %
            % \begin{indt}{\subsubsection{Conclusion}} %{{{3
            %     XXX
            % \end{indt} %}}}3
        \end{indt} %}}}2

        \begin{indt}{\subsection{FR5 -- Authority notification}} %{{{2
            \label{sub:fr5}

            \begin{indt}{\subsubsection{Introduction}} %{{{3
                The purpose of this functional requirement is to notify law enforcement authorities in real time when the system detects either (i) a parking violation (unauthorized parking, expired/missing authorization) or (ii) an authorization cloning event. This capability is essential to ensure enforcement can occur promptly, which increases deterrence and improves the availability of disabled parking spaces for legitimate users.
                
                In this IoT-based smart parking system, notifications must be delivered with low latency and high reliability. The notification chain must also be secure and auditable because it can initiate enforcement actions. Practically, this implies (1) a dependable communication path from the parking node to the backend and (2) a standardized, secure mechanism from the backend to the authority endpoint (e.g., secure REST API, email/SMS fallback, or an event feed).

                TODO: here maybe two tables are needed: one to compare the \emph{protocols} (e.g HTTP, MQTT, CoaP, ...), and one to compare the \emph{authority endpoint} (SMS, Discord, Email, WhatsApp, push notification on smartphone, ad hoc app, ...)
            \end{indt} %}}}3

            \begin{indt}{\subsubsection{Comparison of real-time notification solutions}} %{{{3
                \autoref{tab:fr5_comparison} compares several viable approaches to deliver real-time alerts from the system to authorities.

                \begin{table}[H]% {{{4
                    \centering

                    {\small
                    \begin{tabular}{p{3cm} p{50pt} p{50pt} p{50pt} p{50pt} p{50pt} p{50pt} p{50pt}}
                        \hline
                        \textbf{Technology}
                        & \textbf{Performan\-ce (NF1)}
                        & \textbf{Reliability (NF2)}
                        & \textbf{Durability (NF3)}
                        & \textbf{Resource efficiency (NF5)}
                        & \textbf{Privacy (NF7)}
                        & \textbf{Afforda\-bi\-lity (NF8)}
                        & \textbf{Compati\-bility (NF9)}
                        \\

                        \hline
                        XXX
                        \newline
                        \texttt{ref}
                        & \cellcolor{white} XXX
                        & \cellcolor{white} XXX
                        & \cellcolor{white} XXX
                        & \cellcolor{white} XXX
                        & \cellcolor{white} XXX
                        & \cellcolor{white} XXX
                        & \cellcolor{white} XXX
                        \\

                        XXX
                        \newline
                        \texttt{ref}
                        & \cellcolor{white} XXX
                        & \cellcolor{white} XXX
                        & \cellcolor{white} XXX
                        & \cellcolor{white} XXX
                        & \cellcolor{white} XXX
                        & \cellcolor{white} XXX
                        & \cellcolor{white} XXX
                        \\

                        XXX
                        \newline
                        \texttt{ref}
                        & \cellcolor{white} XXX
                        & \cellcolor{white} XXX
                        & \cellcolor{white} XXX
                        & \cellcolor{white} XXX
                        & \cellcolor{white} XXX
                        & \cellcolor{white} XXX
                        & \cellcolor{white} XXX
                        \\

                        XXX
                        \newline
                        \texttt{ref}
                        & \cellcolor{white} XXX
                        & \cellcolor{white} XXX
                        & \cellcolor{white} XXX
                        & \cellcolor{white} XXX
                        & \cellcolor{white} XXX
                        & \cellcolor{white} XXX
                        & \cellcolor{white} XXX
                        \\

                        \hline
                    \end{tabular}
                    }

                    \caption{Comparison of violation identification mechanisms}
                    \label{tab:fr5_comparison}
                \end{table}% }}}4
            \end{indt} %}}}3

            \begin{indt}{\subsubsection{Conclusion}} %{{{3
                XXX
            \end{indt} %}}}3

            \begin{indt}{\subsubsection{References}} %{{{3
                \begin{indi}[0pt]{}
                    \item XXX: \\
                        \url{https://}
                    \item XXX: \\
                        \url{https://}
                    \item XXX: \\
                        \url{https://}
                \end{indi}
            \end{indt} %}}}3
        \end{indt} %}}}2

        % \begin{indt}{\subsection{FR6 -- Parking availability visualization}} %{{{2
        %     \begin{indt}{\subsubsection{Introduction}} %{{{3
        %         The objective of this functional requirement is to provide users with a clear and real-time visualization of available disabled parking spaces within a selected geographical area. This functionality directly improves accessibility for people with disabilities by reducing the time and effort required to locate an appropriate parking space, especially in dense urban environments.
        %
        %         % Within the IoT-based smart parking system, parking availability visualization serves as the primary user-facing interface. It relies on continuously updated parking spot status data collected from distributed IoT nodes and processed by the centralized backend. Accurate and intuitive visualization is essential to ensure user trust and to enable informed decision-making, particularly when combined with reservation capabilities.
        %     \end{indt} %}}}3
        %
        %     \begin{indt}{\subsubsection{Comparison of visualization and mapping solutions}} %{{{3
        %         Several technical solutions can be used to visualize parking availability data. \autoref{tab:fr6_comparison} presents a comparison of the most relevant mechanisms for this project.
        %
        %         \begin{table}[H]% {{{4
        %             \centering
        %
        %             {\small
        %             \begin{tabular}{p{3cm} p{80pt} p{80pt} p{80pt} p{80pt}}
        %                 \hline
        %                 \textbf{Technology}
        %                 & \textbf{Reliability (NF2)}
        %                 & \textbf{Privacy (NF7)}
        %                 & \textbf{Afforda\-bility (NF8)}
        %                 & \textbf{Compati\-bility (NF9)}
        %                 \\
        %
        %                 \hline
        %                 Phone app
        %                 & \cellcolor{pastelOrange} Depends on network, but can cache data
        %                 & \cellcolor{pastelGreen} Good privacy is possible
        %                 & \cellcolor{pastelRed} Need to maintain multiple app versions
        %                 & \cellcolor{pastelRed} Users need compatible phone
        %                 \\
        %
        %                 Web app
        %                 & \cellcolor{pastelRed} Depends on network's availability
        %                 & \cellcolor{pastelGreen} Good privacy is possible
        %                 & \cellcolor{pastelGreen} Low maintenance cost
        %                 & \cellcolor{pastelGreen} Most users' devices are able to interact with web pages
        %                 \\
        %
        %                 Command line app
        %                 & \cellcolor{pastelOrange} Depends on network, but can cache data
        %                 & \cellcolor{pastelOrange} If using Discord/Telegram, data is shared to that third-party
        %                 & \cellcolor{pastelGreen} Low maintenance cost
        %                 & \cellcolor{pastelOrange} While probably compatible with user's devices, it is not user-friendly
        %                 \\
        %
        %                 \hline
        %             \end{tabular}
        %             }
        %
        %             \caption{Comparison of user interfaces}
        %             \label{tab:fr6_comparison}
        %         \end{table}% }}}4
        %     \end{indt} %}}}3
        %
        %     \begin{indt}{\subsubsection{Conclusion}} %{{{3
        %         In the light of the \autoref{tab:fr6_comparison}, the \emph{web interface} seems the most appropriate for the user interface as it has the largest compatibility.
        %         Regarding reliability, in the event of a network outage, no real time data is possible. Even if with a native application, it might possible to cache data, it will become outdated and the visualisation will be wrong.
        %         We put a stronger emphasis on the compatibility of end user's device.
        %
        %         This interface will also be appropriate to implement the multi-level authorizations (admin, user, ...)
        %     \end{indt} %}}}3
        % \end{indt} %}}}2

        % \begin{indt}{\subsection{FR7 -- Advance reservation}} %{{{2
        %     \begin{indt}{\subsubsection{Introduction}} %{{{3
        %         The objective of this functional requirement is to allow authorized users to reserve a disabled parking space up to one hour in advance. This feature significantly improves accessibility and planning for people with disabilities by reducing uncertainty when traveling to areas with limited parking availability.
        %
        %         In the context of the IoT-based smart parking system, advance reservation must be tightly integrated with real-time parking availability data to avoid conflicts and ensure fairness. The reservation mechanism must prevent double-booking, enforce time constraints, and remain synchronized with on-site occupancy detection in order to release or confirm reservations accurately.
        %     \end{indt} %}}}3
        %
        %     \begin{indt}{\subsubsection{Comparison of reservation management approaches}} %{{{3
        %         Several technical approaches can be used to implement an advance reservation mechanism. \autoref{tab:fr7_comparison} compares the most relevant solutions for this project.
        %
        %         \begin{table}[H]% {{{4
        %             \centering
        %
        %             {\small
        %             \begin{tabular}{p{3cm} p{50pt} p{50pt} p{50pt} p{50pt} p{50pt} p{50pt} p{50pt}}
        %                 \hline
        %                 \textbf{Technology}
        %                 & \textbf{Performan\-ce (NF1)}
        %                 & \textbf{Reliability (NF2)}
        %                 & \textbf{Durability (NF3)}
        %                 & \textbf{Resource efficiency (NF5)}
        %                 & \textbf{Privacy (NF7)}
        %                 & \textbf{Afforda\-bi\-lity (NF8)}
        %                 & \textbf{Compati\-bility (NF9)}
        %                 \\
        %
        %                 \hline
        %                 XXX
        %                 \newline
        %                 \texttt{ref}
        %                 & \cellcolor{white} XXX
        %                 & \cellcolor{white} XXX
        %                 & \cellcolor{white} XXX
        %                 & \cellcolor{white} XXX
        %                 & \cellcolor{white} XXX
        %                 & \cellcolor{white} XXX
        %                 & \cellcolor{white} XXX
        %                 \\
        %
        %                 XXX
        %                 \newline
        %                 \texttt{ref}
        %                 & \cellcolor{white} XXX
        %                 & \cellcolor{white} XXX
        %                 & \cellcolor{white} XXX
        %                 & \cellcolor{white} XXX
        %                 & \cellcolor{white} XXX
        %                 & \cellcolor{white} XXX
        %                 & \cellcolor{white} XXX
        %                 \\
        %
        %                 XXX
        %                 \newline
        %                 \texttt{ref}
        %                 & \cellcolor{white} XXX
        %                 & \cellcolor{white} XXX
        %                 & \cellcolor{white} XXX
        %                 & \cellcolor{white} XXX
        %                 & \cellcolor{white} XXX
        %                 & \cellcolor{white} XXX
        %                 & \cellcolor{white} XXX
        %                 \\
        %
        %                 XXX
        %                 \newline
        %                 \texttt{ref}
        %                 & \cellcolor{white} XXX
        %                 & \cellcolor{white} XXX
        %                 & \cellcolor{white} XXX
        %                 & \cellcolor{white} XXX
        %                 & \cellcolor{white} XXX
        %                 & \cellcolor{white} XXX
        %                 & \cellcolor{white} XXX
        %                 \\
        %
        %                 \hline
        %             \end{tabular}
        %             }
        %
        %             \caption{Comparison of violation identification mechanisms}
        %             \label{tab:fr7_comparison}
        %         \end{table}% }}}4
        %     \end{indt} %}}}3
        %
        %     \begin{indt}{\subsubsection{Conclusion}} %{{{3
        %         XXX
        %     \end{indt} %}}}3
        % \end{indt} %}}}2

        \begin{indt}{\subsection{FR8 -- Reservation timeout management}} %{{{2
            \begin{indt}{\subsubsection{Introduction}} %{{{3
                The objective of this functional requirement is to automatically cancel a parking space reservation if the reserved vehicle does not arrive within a predefined grace period. This functionality is essential to ensure efficient utilization of disabled parking spaces and to prevent situations where a space remains unused due to unfulfilled reservations.

                In the context of the IoT-based smart parking system, reservation timeout management must operate reliably and autonomously by correlating reservation timestamps with real-time occupancy detection data. By enforcing a grace period, the system balances flexibility for users with the need to maximize availability and fairness for other authorized drivers.
            \end{indt} %}}}3

            \begin{indt}{\subsubsection{Comparison of reservation timeout approaches}} %{{{3
                Several technical approaches can be used to manage reservation timeout. \autoref{tab:fr8_comparison} compares the most relevant solutions for this project.

                \begin{table}[H]% {{{4
                    \centering

                    {\small
                    \begin{tabular}{p{3cm} p{50pt} p{50pt} p{50pt} p{50pt} p{50pt} p{50pt} p{50pt}}
                        \hline
                        \textbf{Technology}
                        & \textbf{Performan\-ce (NF1)}
                        & \textbf{Reliability (NF2)}
                        & \textbf{Durability (NF3)}
                        & \textbf{Resource efficiency (NF5)}
                        & \textbf{Privacy (NF7)}
                        & \textbf{Afforda\-bi\-lity (NF8)}
                        & \textbf{Compati\-bility (NF9)}
                        \\

                        \hline
                        XXX
                        \newline
                        \texttt{ref}
                        & \cellcolor{white} XXX
                        & \cellcolor{white} XXX
                        & \cellcolor{white} XXX
                        & \cellcolor{white} XXX
                        & \cellcolor{white} XXX
                        & \cellcolor{white} XXX
                        & \cellcolor{white} XXX
                        \\

                        XXX
                        \newline
                        \texttt{ref}
                        & \cellcolor{white} XXX
                        & \cellcolor{white} XXX
                        & \cellcolor{white} XXX
                        & \cellcolor{white} XXX
                        & \cellcolor{white} XXX
                        & \cellcolor{white} XXX
                        & \cellcolor{white} XXX
                        \\

                        XXX
                        \newline
                        \texttt{ref}
                        & \cellcolor{white} XXX
                        & \cellcolor{white} XXX
                        & \cellcolor{white} XXX
                        & \cellcolor{white} XXX
                        & \cellcolor{white} XXX
                        & \cellcolor{white} XXX
                        & \cellcolor{white} XXX
                        \\

                        XXX
                        \newline
                        \texttt{ref}
                        & \cellcolor{white} XXX
                        & \cellcolor{white} XXX
                        & \cellcolor{white} XXX
                        & \cellcolor{white} XXX
                        & \cellcolor{white} XXX
                        & \cellcolor{white} XXX
                        & \cellcolor{white} XXX
                        \\

                        \hline
                    \end{tabular}
                    }

                    \caption{Comparison of reservation timeout approaches}
                    \label{tab:fr8_comparison}
                \end{table}% }}}4
            \end{indt} %}}}3

            \begin{indt}{\subsubsection{Conclusion}} %{{{3
                XXX
            \end{indt} %}}}3

            \begin{indt}{\subsubsection{References}} %{{{3
                \begin{indi}[0pt]{}
                    \item XXX: \\
                        \url{https://}
                    \item XXX: \\
                        \url{https://}
                    \item XXX: \\
                        \url{https://}
                \end{indi}
            \end{indt} %}}}3
        \end{indt} %}}}2

        \begin{indt}{\subsection{FR9 -- Parking spot status display}} %{{{2
            \begin{indt}{\subsubsection{Introduction}} %{{{3
                The objective of this functional requirement is to display the current status of a disabled parking space, namely whether it is free, reserved, or currently occupied (parked). This information is essential for both end users and system operators, as it provides immediate feedback on parking availability and reservation enforcement.

                Within the IoT-based smart parking system, parking spot status acts as a synthesis of multiple data sources, including real-time sensor data from the parking node and reservation information managed by the backend. A clear and reliable status representation improves user understanding, supports decision-making, and ensures consistency between physical occupancy and digital system state.

                TODO: also send the current state to the platform. So need to compare the different \emph{communication technologies} also.
            \end{indt} %}}}3

            \begin{indt}{\subsubsection{Comparison of parking spot status display mechanisms}} %{{{3
                Several approaches can be used to determine and display the parking spot status
                \autoref{tab:fr9_comparison} compares the most relevant solutions for this project.

                \begin{table}[H]% {{{4
                    \centering

                    {\small
                    \begin{tabular}{p{3cm} p{50pt} p{50pt} p{50pt} p{50pt} p{50pt} p{50pt} p{50pt}}
                        \hline
                        \textbf{Technology}
                        & \textbf{Performan\-ce (NF1)}
                        & \textbf{Reliability (NF2)}
                        & \textbf{Durability (NF3)}
                        & \textbf{Resource efficiency (NF5)}
                        & \textbf{Privacy (NF7)}
                        & \textbf{Afforda\-bi\-lity (NF8)}
                        & \textbf{Compati\-bility (NF9)}
                        \\

                        \hline
                        XXX
                        \newline
                        \texttt{ref}
                        & \cellcolor{white} XXX
                        & \cellcolor{white} XXX
                        & \cellcolor{white} XXX
                        & \cellcolor{white} XXX
                        & \cellcolor{white} XXX
                        & \cellcolor{white} XXX
                        & \cellcolor{white} XXX
                        \\

                        XXX
                        \newline
                        \texttt{ref}
                        & \cellcolor{white} XXX
                        & \cellcolor{white} XXX
                        & \cellcolor{white} XXX
                        & \cellcolor{white} XXX
                        & \cellcolor{white} XXX
                        & \cellcolor{white} XXX
                        & \cellcolor{white} XXX
                        \\

                        XXX
                        \newline
                        \texttt{ref}
                        & \cellcolor{white} XXX
                        & \cellcolor{white} XXX
                        & \cellcolor{white} XXX
                        & \cellcolor{white} XXX
                        & \cellcolor{white} XXX
                        & \cellcolor{white} XXX
                        & \cellcolor{white} XXX
                        \\

                        XXX
                        \newline
                        \texttt{ref}
                        & \cellcolor{white} XXX
                        & \cellcolor{white} XXX
                        & \cellcolor{white} XXX
                        & \cellcolor{white} XXX
                        & \cellcolor{white} XXX
                        & \cellcolor{white} XXX
                        & \cellcolor{white} XXX
                        \\

                        \hline
                    \end{tabular}
                    }

                    \caption{Comparison of parking spot status display}
                    \label{tab:fr9_comparison}
                \end{table}% }}}4
            \end{indt} %}}}3

            \begin{indt}{\subsubsection{Conclusion}} %{{{3
                XXX
            \end{indt} %}}}3

            \begin{indt}{\subsubsection{References}} %{{{3
                \begin{indi}[0pt]{}
                    \item XXX: \\
                        \url{https://}
                    \item XXX: \\
                        \url{https://}
                    \item XXX: \\
                        \url{https://}
                \end{indi}
            \end{indt} %}}}3
        \end{indt} %}}}2

        % \begin{indt}{\subsection{FR10 -- User registration}} %{{{2
        %     \begin{indt}{\subsubsection{Introduction}} %{{{3
        %         The objective of this functional requirement is to allow users and administrators to register on the smart parking platform. User registration is a prerequisite for accessing personalized services such as parking availability visualization, advance reservation, and parking session management. For administrators, registration enables system configuration, monitoring, and enforcement oversight.
        %
        %         In the context of an IoT-based smart parking system dedicated to people with disabilities, the registration process must be secure, reliable, and compliant with data protection regulations. It must also support role differentiation (user versus administrator) and integrate seamlessly with backend services that manage authorizations, reservations, and notifications.
        %     \end{indt} %}}}3
        %
        %     \begin{indt}{\subsubsection{Comparison of user registration and authentication solutions}} %{{{3
        %         Several technical approaches can be used to implement user registration and account management.
        %         \autoref{tab:fr10_comparison} compares the most relevant solutions for this project.
        %
        %         \begin{table}[H]% {{{4
        %             \centering
        %
        %             {\small
        %             \begin{tabular}{p{3cm} p{50pt} p{50pt} p{50pt} p{50pt} p{50pt} p{50pt} p{50pt}}
        %                 \hline
        %                 \textbf{Technology}
        %                 & \textbf{Performan\-ce (NF1)}
        %                 & \textbf{Reliability (NF2)}
        %                 & \textbf{Durability (NF3)}
        %                 & \textbf{Resource efficiency (NF5)}
        %                 & \textbf{Privacy (NF7)}
        %                 & \textbf{Afforda\-bi\-lity (NF8)}
        %                 & \textbf{Compati\-bility (NF9)}
        %                 \\
        %
        %                 \hline
        %                 XXX
        %                 \newline
        %                 \texttt{ref}
        %                 & \cellcolor{white} XXX
        %                 & \cellcolor{white} XXX
        %                 & \cellcolor{white} XXX
        %                 & \cellcolor{white} XXX
        %                 & \cellcolor{white} XXX
        %                 & \cellcolor{white} XXX
        %                 & \cellcolor{white} XXX
        %                 \\
        %
        %                 XXX
        %                 \newline
        %                 \texttt{ref}
        %                 & \cellcolor{white} XXX
        %                 & \cellcolor{white} XXX
        %                 & \cellcolor{white} XXX
        %                 & \cellcolor{white} XXX
        %                 & \cellcolor{white} XXX
        %                 & \cellcolor{white} XXX
        %                 & \cellcolor{white} XXX
        %                 \\
        %
        %                 XXX
        %                 \newline
        %                 \texttt{ref}
        %                 & \cellcolor{white} XXX
        %                 & \cellcolor{white} XXX
        %                 & \cellcolor{white} XXX
        %                 & \cellcolor{white} XXX
        %                 & \cellcolor{white} XXX
        %                 & \cellcolor{white} XXX
        %                 & \cellcolor{white} XXX
        %                 \\
        %
        %                 XXX
        %                 \newline
        %                 \texttt{ref}
        %                 & \cellcolor{white} XXX
        %                 & \cellcolor{white} XXX
        %                 & \cellcolor{white} XXX
        %                 & \cellcolor{white} XXX
        %                 & \cellcolor{white} XXX
        %                 & \cellcolor{white} XXX
        %                 & \cellcolor{white} XXX
        %                 \\
        %
        %                 \hline
        %             \end{tabular}
        %             }
        %
        %             \caption{Comparison of user registration and authentication solutions}
        %             \label{tab:fr10_comparison}
        %         \end{table}% }}}4
        %     \end{indt} %}}}3
        %
        %     \begin{indt}{\subsubsection{Conclusion}} %{{{3
        %         XXX
        %     \end{indt} %}}}3
        % \end{indt} %}}}2
    \end{indt} %}}}1

    \newpage

    \begin{indt}{\section{Design}} %{{{1
        \begin{indt}{\subsection{Selected Hardware and Technologies}} %{{{2
            \begin{table}[H]% {{{3
                \centering
            
                \begin{tabular}{|l p{9cm}|}
                    \hline

                    \textbf{Technology}
                    & \textbf{Choice}
                    \\

                    \hline
                    \hline
                    Microcontroller
                    & NodeMCU ESP8266 (integrated Wi-Fi, low power)
                    \\

                    \hline
                    Occupancy sensor
                    & Ultrasonic sensor
                    \\

                    \hline
                    Authorization
                    & RFID reader (Mifare Classic)
                    \\
                    
                    \hline
                    Connectivity
                    & Wi-Fi
                    \\

                    \hline
                    Backend
                    & TODO
                    \\
                    
                    \hline
                    Frontend
                    & TODO
                    \\

                    \hline
                \end{tabular}
            
                \caption{Selected hardware and technologies}
                \label{tab:selected}
            \end{table}% }}}3
        \end{indt} %}}}2
    \end{indt} %}}}1

    \begin{indt}{\section{Conclusion}} %{{{1
        This IoT-based smart parking system addresses a real societal need by improving accessibility for people with disabilities while providing authorities with efficient tools to prevent abuse.
        The proposed architecture is scalable, energy-efficient, and compliant with security and privacy requirements, making it suitable for real-world deployment.
    \end{indt} %}}}1

    \begin{indt}{\section{Annexes}} %{{{1
        \begin{itemize}[itemsep=0pt, leftmargin=\leftskip+12pt]
            \item NodeMCU ESP8266 Datasheet: 
                \url{https://cdn-shop.adafruit.com/product-files/2471/0A-ESP8266__Datasheet__EN_v4.3.pdf}

            \item MFRC522 RFID Reader Datasheet (NXP): 
                \url{https://www.nxp.com/docs/en/data-sheet/MFRC522.pdf}

            \item HC-SR04 Ultrasonic Distance Sensor Datasheet: 
                \url{https://cdn.sparkfun.com/datasheets/Sensors/Proximity/HCSR04.pdf}

        \end{itemize}
    \end{indt} %}}}1
    
\end{document}
%--------------------------------------------End

% vim:foldmethod=marker:foldlevel=0
